\documentclass[12pt]{article}
\usepackage[english]{babel}
\usepackage{natbib}
\usepackage{url}
\usepackage[utf8x]{inputenc}
\usepackage{amsmath}
\usepackage{graphicx}
\graphicspath{{images/}}
\usepackage{parskip}
\usepackage{fancyhdr}
\usepackage{vmargin}
\usepackage{amsmath}
\usepackage{mathtools}
\usepackage{hyperref}
\usepackage{xcolor}
\usepackage{float}
\usepackage{siunitx}
\usepackage{rotating}
\usepackage{multirow}
\usepackage{listings}
\usepackage{hyperref}
\setmarginsrb{3 cm}{2.5 cm}{3 cm}{2.5 cm}{1 cm}{1.5 cm}{1 cm}{1.5 cm}

\title{Tutorial 2: C-Based High Level Synthesis (HLS) Design Flow}		% Title
\author{Jianwen Zhu}								% Author
\date{\today}									% Date
\newcommand{\course} {ECE1769H}
\newcommand{\quotes}[1]{``#1''}

\makeatletter
\let\thetitle\@title
\let\theauthor\@author
\let\thedate\@date
\makeatother

\pagestyle{fancy}
\fancyhf{}
\lhead{\course}
\rhead{\thetitle}
\cfoot{\thepage}

\newcommand {\comment}[1]{\iffalse #1 \fi}

\begin{document}

%%%%%%%%%%%%%%%%%%%%%%%%%%%%%%%%%%%%%%%%%%%%%%%%%%%%%%%%%%%%%%%%%%%%%%%%%%%%%%%%%%%%%%%%%

\begin{titlepage}
	\centering
    \vspace*{0.5 cm}
    \includegraphics[scale = 0.12]{UofT.png}\\[1.0 cm]	% University Logo
    \textsc{\LARGE University of Toronto}\\[2.0 cm]	% University Name
	\textsc{\Large \course}\\[0.5 cm]				% Course Code
	\textsc{\large Behavioral Synthesis of Digital Integrated Circuits}\\[0.5 cm]				% Course Name
	\rule{\linewidth}{0.2 mm} \\[0.4 cm]
	{ \huge \bfseries \thetitle}\\
	\rule{\linewidth}{0.2 mm} \\[1.5 cm]
	
	%\begin{minipage}{0.4\textwidth}
		%\begin{flushleft} 
		\large
			\emph{Instructor:}\\
			\theauthor 
		%	\end{flushleft}
		%	\end{minipage}~
			
 
	\vfill
	
\end{titlepage}

%%%%%%%%%%%%%%%%%%%%%%%%%%%%%%%%%%%%%%%%%%%%%%%%%%%%%%%%%%%%%%%%%%%%%%%%%%%%%%%%%%%%%%%%%
\pagebreak

%%%%%%%%%%%%%%%%%%%%%%%%%%%%%%%%%%%%%%%%%%%%%%%%%%%%%%%%%%%%%%%%%%%%%%%%%%%%%%%%%%%%%%%%%

\section{Introduction}

In this tutorial, you will learn C-based High Level Synthesis (HLS)
design flow.

You start with a reference application in C, where you define the
functionality (behavior) of your design. There are two alternative
path to bring the design down to the register transfer level:

\begin{enumerate}
\item Register Transfer Level (RTL) design flow, where you enter your
  design at RTL in Verilog {\bf manually}, which is covered in the
  previous tutorial;
  
\item C-based design flow, where you enter your design in C, and then
  use high level synthesis (HLS) to generate a RTL design {\bf
    automatically}.
\end{enumerate}

In both design flows, you can use {\bf compiled functional simulation} to
verify your design at the register transfer level, which is covered in the
previous design flow.

Three primary EDA tools are used in this tutorial:

\begin{enumerate}
\item Verilator, an open-sourced Verilog compiler, which convert the
  Verilog into C++. Together with a C++ testbench, the RTL design can
  be verified in a cycle-accurate fashion.  C-based simulation reduces
  the simulation time significantly compared to the traditional
  discrete-event simulation.

\item GTKWave, an open-sourced waveform viewer, can be used to inspect
  waveforms from the result of verilator simulation.
  
\item Vivada HLS, a commercial HLS tool from Xilinx, is used to convert
  C/C++ design into RTL in Verilog.

\end{enumerate}


\section{Environment Setup}

The required files for this tutorial are available for download
through the following command:

\begin{lstlisting}[language=bash]
  $git clone https://github.com/jianwen-zhu/2024-fall-ece1749-jzhu.git
\end{lstlisting}

After cloning the material, you need to create a “site.mk” based on
“site.mk.sample” and provide correct path for the required softwares.

\begin{lstlisting}[language=bash]
  $cd 2024-fall-ece1749-jzhu
  $cd tutorial
  $cp site.mk.sample site.mk
\end{lstlisting}


All the required softwares required for this tutorial are already
installed on UG machines and you do not need to install anything if
you make your working directory on those machines. We STRONGLY
RECOMMEND that you do the clone on UG machines rather than your local
machine. In that case the variables need to be setup as follows:

\begin{lstlisting}[language=bash]
VIVADO_HLS_PATH = /cad2/ece1769s/xilinx/Vivado_HLS/2014.4/bin
VERILATOR_PATH = /cad2/ece1769s/verilator/bin
GTKWAVE_PATH = /cad2/ece1769s/gtkwave/bin
\end{lstlisting}

NOTE: if you decide to work on UG machines through SSH, you need to
enable -X option since Gtkwave opens a GUI.

\section{Example Design}

In this tutorial we pick an example design and walk you through the
steps of both flows. Our case study is the dot product
calculator. Given two input vectors, the dot product is calculated
using the following formula:

\[
A \cdot B = \Sigma_{i=1}^{n} A_i B_i
\]

The design has a {\em start} and {\em done} signal. Upon receiving the
{\em start} signal, the hardware reads the A and B vectors from the
memory and writes their dot product result to the memory at the end of
the calculation, after which the {\em done} signal is asserted.

\section{Directory Organization}

A directory called “tutorial” is provided for you and it contains the
following directories and files:

You have seen the following in Tutorial 1.

\begin{enumerate}

\item ref: Reference design in C along with its makefile to be used
    as the golden model;

\item dot-rtl1: The Verilog Implementation of dot product unit;

\item tb: The C-based testbench to simulate the RTL design;

\item doc: Reference manuals for relevant CAD tools;

\item common.mk: Makefile rules that encodes the design flow;

\item site.mk.sample: Site-specific Makefile configurations (NOTE: You
  need to copy this file into “site.mk” and define correct paths for
  tools).

\end{enumerate}

The new directory that you should focus your attention to is:

\begin{enumerate}
\item dot-c: C (HLS) implementation of dot product along with a
  Makefile to perform synthesis in Vivado, and simulation in
  Verilator.
\end{enumerate}

\section{Vivado HLS}

Compared to the manual RTL design flow covered in Tutorial 1,
the only additional step we have in the HLS flow is to use
the Vivado HLS to generate the RTL design automatically.

Go to the “dot-c” directory:

\begin{lstlisting}[language=bash]
  $cd dot-c
\end{lstlisting}

You should find the following files:

\begin{enumerate}
\item dot\_product.c and dot\_product.h: C and header file for the design
  made suitable for HLS;

\item dot\_product\_tb.c: C testbench to test the design;

\item directives.tcl: Vivado HLS directives;

\item script.tcl: TCL script to run Vivado HLS;

\item Makefile: The project specific Makefile that includes the “common.mk” in
  the parent directory.
  
\end{enumerate}

Type: 
\begin{lstlisting}[language=bash]
  $make csyn
\end{lstlisting}

This make target uses “script.tcl” to generate Vivado HLS project
using the provided source files. It also synthesizes the design and
generates RTL Verilog. In Vivado HLS you can use certain directives to
improve the generated hardware. these directives should be placed in
the “directives.tcl” file. However, the detail is beyond the scope of
this tutorial. You can refer to Vivado HLS manuals if you are
interested. The generated RTL design will be found in
“c/vivado\_hls/sol/syn/verilog” directory. The final output RTL is
copied into the working directory.

\section{Verifying RTL Design}

You can use the same procedure in Tutorial 1 to verify the
synthesized RTL design.

In the same directory, do: 

\begin{lstlisting}[language=bash]
  $make verilate
  $make sim
\end{lstlisting}

And do: 

\begin{lstlisting}[language=bash]
  $make view
\end{lstlisting}

to view the waveform.


\section{Exercises}

The design we provided for you is (intentionally) sub-optimal and
there are many potentials for improvement. You can explore the Vivada
manual, especially the pragmas and synthesis directivies, to come up
with a better design.

\end{document}
