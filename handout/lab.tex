\documentclass[12pt]{article}
\usepackage[english]{babel}
\usepackage{natbib}
\usepackage{url}
%\usepackage[utf8x]{inputenc}
\usepackage{amsmath}
\usepackage{graphicx}
\graphicspath{{images/}}
\usepackage{parskip}
\usepackage{fancyhdr}
\usepackage{vmargin}
\usepackage{amsmath}
%\usepackage{mathtools}
\usepackage{hyperref}
\usepackage{xcolor}
\usepackage{float}
%\usepackage{siunitx}
\usepackage{rotating}
\usepackage{multirow}
\usepackage{listings}
\usepackage{hyperref}
\setmarginsrb{3 cm}{2.5 cm}{3 cm}{2.5 cm}{1 cm}{1.5 cm}{1 cm}{1.5 cm}

\title{Lab: Dot Product Accelerator Design}			% Title
\author{Jianwen Zhu}						% Author
\date{\today}							% Date
\newcommand{\course} {ECE1769H}
\newcommand{\quotes}[1]{``#1''}

\makeatletter
\let\thetitle\@title
\let\theauthor\@author
\let\thedate\@date
\makeatother

\pagestyle{fancy}
\fancyhf{}
\lhead{\course}
\rhead{\thetitle}
\cfoot{\thepage}

\newcommand {\comment}[1]{\iffalse #1 \fi}

\begin{document}

%%%%%%%%%%%%%%%%%%%%%%%%%%%%%%%%%%%%%%%%%%%%%%%%%%%%%%%%%%%%%%%%%%%%%%%%%%%%%%%%%%%%%%%%%

\begin{titlepage}
	\centering
    \vspace*{0.5 cm}
%    \includegraphics[scale = 0.12]{images/UofT.png}\\[1.0 cm]	% University Logo
    \textsc{\LARGE University of Toronto}\\[2.0 cm]	% University Name
	\textsc{\Large \course}\\[0.5 cm]				% Course Code
	\textsc{\large Behavioral Synthesis of Digital Integrated Circuits}\\[0.5 cm]				% Course Name
	\rule{\linewidth}{0.2 mm} \\[0.4 cm]
	{ \huge \bfseries \thetitle}\\
	\rule{\linewidth}{0.2 mm} \\[1.5 cm]
	
	%\begin{minipage}{0.4\textwidth}
		%\begin{flushleft} 
		\large
			\emph{Instructor:}\\
			\theauthor 
		%	\end{flushleft}
		%	\end{minipage}~
			
 
	\vfill
	
\end{titlepage}

%%%%%%%%%%%%%%%%%%%%%%%%%%%%%%%%%%%%%%%%%%%%%%%%%%%%%%%%%%%%%%%%%%%%%%%%%%%%%%%%%%%%%%%%%
\pagebreak

%%%%%%%%%%%%%%%%%%%%%%%%%%%%%%%%%%%%%%%%%%%%%%%%%%%%%%%%%%%%%%%%%%%%%%%%%%%%%%%%%%%%%%%%%

\section{Overview}

In this lab you are challenged to build an accelerator for
the dot product kernel algorithm.

\section{Prerequsite}

You should complete the following tutorials on your own:

\begin{enumerate}
  \item Tutorial 1: Register Transfer Level (RTL) Design Flow;
  \item Tutorial 2: C-Based High Level Synthesis (HLS) Design Flow;
  \item Tutorial 3: Logic Synthesis Design Flow Using YOSYS.
\end{enumerate}

\section{Requirements}

You are required to use {\em either} the RTL flow {\em or} the HLS
flow to improve the dot product design.  The build system illustrated
in Tutorial 1 and 2 should be utilized for synthesis and
verification. It is required that the Verilog design is simulated
(using Verilator) with the same test cases as those in the reference
code.

Logic synthesis should be carried out using the build flow in
Tutorial3. The design statistics such as gate count and maximum
frequency should be gathered. An optional, but bonus effort is the use
of optimization techniques at the logic level.

\section{Deliverables}

You are required to deliver:

\begin{itemize}
  \item The source code of the completed design packaged in tar.gz format;
  \item A technical report in PDF format. 
\end{itemize}

The source code should have the similar directory structure as the
sample given in the tutorials. In particular, it should contain the
Makefile with targets that can re-produce all the required results,
including those included in the report.

The technical report should include the following material:

\begin{itemize}
  \item Description of the techniques and decisions you made to improve the design;
   \item Description of design process and experience;
  \item Description of performance measurement results;
  \item Conclusion: a discussion of lesson learned and
        future work.
\end{itemize}

\section{Due Date}

The deliverables are due on December 24th. No marks can be given
before all deliverables are submitted.

\end{document}
