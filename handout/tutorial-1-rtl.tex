\documentclass[12pt]{article}
\usepackage[english]{babel}
\usepackage{natbib}
\usepackage{url}
\usepackage[utf8x]{inputenc}
\usepackage{amsmath}
\usepackage{graphicx}
\graphicspath{{images/}}
\usepackage{parskip}
\usepackage{fancyhdr}
\usepackage{vmargin}
\usepackage{amsmath}
\usepackage{mathtools}
\usepackage{hyperref}
\usepackage{xcolor}
\usepackage{float}
\usepackage{siunitx}
\usepackage{rotating}
\usepackage{multirow}
\usepackage{listings}
\usepackage{hyperref}
\setmarginsrb{3 cm}{2.5 cm}{3 cm}{2.5 cm}{1 cm}{1.5 cm}{1 cm}{1.5 cm}

\title{Tutorial 1: Register Transfer Level (RTL) Design Flow}			% Title
\author{Jianwen Zhu}								% Author
\date{\today}									% Date
\newcommand{\course} {ECE1769H}
\newcommand{\quotes}[1]{``#1''}

\makeatletter
\let\thetitle\@title
\let\theauthor\@author
\let\thedate\@date
\makeatother

\pagestyle{fancy}
\fancyhf{}
\lhead{\course}
\rhead{\thetitle}
\cfoot{\thepage}

\newcommand {\comment}[1]{\iffalse #1 \fi}

\begin{document}

%%%%%%%%%%%%%%%%%%%%%%%%%%%%%%%%%%%%%%%%%%%%%%%%%%%%%%%%%%%%%%%%%%%%%%%%%%%%%%%%%%%%%%%%%

\begin{titlepage}
	\centering
    \vspace*{0.5 cm}
    \includegraphics[scale = 0.12]{UofT.png}\\[1.0 cm]	% University Logo
    \textsc{\LARGE University of Toronto}\\[2.0 cm]	% University Name
	\textsc{\Large \course}\\[0.5 cm]				% Course Code
	\textsc{\large Behavioral Synthesis of Digital Integrated Circuits}\\[0.5 cm]				% Course Name
	\rule{\linewidth}{0.2 mm} \\[0.4 cm]
	{ \huge \bfseries \thetitle}\\
	\rule{\linewidth}{0.2 mm} \\[1.5 cm]
	
	%\begin{minipage}{0.4\textwidth}
		%\begin{flushleft} 
		\large
			\emph{Instructor:}\\
			\theauthor 
		%	\end{flushleft}
		%	\end{minipage}~
			
 
	\vfill
	
\end{titlepage}

%%%%%%%%%%%%%%%%%%%%%%%%%%%%%%%%%%%%%%%%%%%%%%%%%%%%%%%%%%%%%%%%%%%%%%%%%%%%%%%%%%%%%%%%%
\pagebreak

%%%%%%%%%%%%%%%%%%%%%%%%%%%%%%%%%%%%%%%%%%%%%%%%%%%%%%%%%%%%%%%%%%%%%%%%%%%%%%%%%%%%%%%%%

\section{Introduction}

In this tutorial, we will learn the {\bf manual} Register Transfer Level
(RTL) design flow of digital design. Specifically, we:


\begin{enumerate}
\item Start with a reference application in {\bf C}, where you define the
functionality (behavior) of your design;
  
\item Manually enter RTL design in {\bf Verilog};
  
\item Verify RTL design using {\bf compiled simulation}.
\end{enumerate}

Two EDA tools are used in this tutorial:

\begin{enumerate}
\item {\bf Verilator}, an open-sourced Verilog compiler, which convert
  the Verilog into C++. Together with a C++ testbench, the RTL design
  can be verified in a cycle-accurate fashion. Compared to the
  traditional discrete-event simulation, compiled simulation reduces
  verification time significantly.
  

\item {\bf GtkWave}, an open-sourced waveform viewer, can be used to
  inspect waveforms from the result of compiled simulation run, and
  debug your design if there is a problem.
  
\end{enumerate}


\section{Environment Setup}

The required files for this tutorial are available for download
through the following command:

\begin{lstlisting}[language=bash]
  $git clone https://github.com/jianwen-zhu/2024-fall-ece1769-jzhu.git
\end{lstlisting}

After cloning the material, you need to create a “site.mk” based on
“site.mk.sample” and provide correct path for the required softwares.

\begin{lstlisting}[language=bash]
  $cd 2024-fall-ece1749-jzhu
  $cd tutorial
  $cp site.mk.sample site.mk
\end{lstlisting}

All the required softwares required for this tutorial are already
installed on UG machines. We STRONGLY RECOMMEND that you do the clone
on UG machines rather than your local machine. In that case the
variables need to be setup as follows:

\begin{lstlisting}[language=bash]
VERILATOR_PATH = /cad2/ece1769s/verilator/bin
GTKWAVE_PATH = /cad2/ece1769s/gtkwave/bin
\end{lstlisting}

NOTE: if you decide to work on UG machines through SSH, you need to
enable -X option since Gtkwave opens a GUI.

\section{Example Design}

In this tutorial we pick an example design and walk you through the
steps of both flows. Our case study is the dot product
calculator. Given two input vectors, the dot product is calculated
using the following formula:

\[
A \cdot B = \Sigma_{i=1}^{n} A_i B_i
\]

The design has a {\em start} and {\em done} signal. Upon receiving the
{\em start} signal, the hardware reads the A and B vectors from the
memory and writes the sum of the product of their respective elements
to the memory, after which the {\em done} signal is asserted.

\section{Directory Organization}

A directory called “tutorial” is provided for you and it contains the
following directories and files:

\begin{enumerate}

\item ref: Reference design in C along with its makefile to be used
    as the golden model;

\item dot-rtl1: The Verilog Implementation of dot product unit;

\item tb: The C-based testbench to simulate the RTL design;

\item doc: Reference manuals for relevant CAD tools;

\item common.mk: Makefile rules that encodes the design flow;

\item site.mk.sample: Site-specific Makefile configurations (NOTE: You
  need to copy this file into “site.mk” and define correct paths for
  tools).

\end{enumerate}

\section{Reference Design in C}

A software implementation of the dot product is provided for you as a
reference in “tutorial/ref” directory. You can compile and run it
using the makefile in that directory. We provide a fixed random input
to the system and you should compare the software implementation’s
output with the hardware’s output that you get in later steps of this
tutorial.

\section{Manual Verilog Implementation}

One Verilog Implementation of dot-product is provided for you in the
“dot-rtl1” directory. For the purpose of testing, we put a block ram
beside the dot product calculator and wrap them as single module. Blow
is the explanation of the files in this directory:

\begin{enumerate}
  
\item datapath.v: Data path module for dot product unit.
  
\item controller.v: Controller module for dot product unit.
  
\item dot\_product.v: The dot product unit that connects the controller
  and the data path.
  
\item Makefile: The project specific Makefile that includes the
  “common.mk” in the parent directory.

\end{enumerate}

\section{C-based Simulation with Verilator}

\subsection{Compiling RTL Design}

The first step towards simulation with Verilator is to convert the
Verilog design into C++. Go to dot-rtl1 Directory:

\begin{lstlisting}[language=bash]
cd dot-rtl1
\end{lstlisting}

Then do:

\begin{lstlisting}[language=bash]
make verilate
\end{lstlisting}

This make target perform several actions:

\begin{enumerate}
\item Generate top-level simulation verilog files:
  sim\_top\_level.v, sim\_dp\_bram.v, sim\_fifo.v, which
  represents the {\em test environment} around the design
  under test (DUT);

\item Generate C++ from the verilog files using Verilator;

\item Link the generated C++ again the C++ testbench in
  the tutorial/tb directory. 
  
\end{enumerate}

Open “tb/sim\_main.cpp” and try to understand how it works.  Verilator
generates a C++ class out of the the top level Verilog module
(sim\_top\_level.v).  In sim\_main.cpp, we instantiate an object from
the said top level class first. We then have access to module ports
through that object. We also instantiate an object from Verilated-VcdC
class to dump the trace into a vcd file. In the testbench, we feed the
{\em start} signal and wait for the {\em done} signal.

Completing this step should generate an executable ready for execution.

\subsection{Running Compiled Simulation}

Do:

\begin{lstlisting}[language=bash]
make sim
\end{lstlisting}

This should run the generated executable from the previous step with
the right settings. If completed successfully, this step should
generate a vcd file that contains the waveform trace.

\subsection{Inspecting Waveforms}

After doing the simulation, you can check the waveforms to make
sure that the design performs as expected.

Do: 

\begin{lstlisting}[language=bash]
  $make view
\end{lstlisting}

to view the waveform. This should open the GtkWave graphical user interface (GUI)
and visualize the contents of the vcd file.

\section{Exercises}

The design we provided for you is (intentionally) sub-optimal and
there are many potentials for improvement. This tutorial can only be
used as a start point for your own project. You are encouraged to
come up with a better design, for example, by cloning dot-rtl1 to
another directory, for example, dot-rtl2, and start from there.

\end{document}
