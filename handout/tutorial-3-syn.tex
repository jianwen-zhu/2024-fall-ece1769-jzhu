\documentclass[12pt]{article}
\usepackage[english]{babel}
\usepackage{natbib}
\usepackage{url}
\usepackage[utf8x]{inputenc}
\usepackage{amsmath}
\usepackage{graphicx}
\graphicspath{{images/}}
\usepackage{parskip}
\usepackage{fancyhdr}
\usepackage{vmargin}
\usepackage{amsmath}
\usepackage{mathtools}
\usepackage{hyperref}
\usepackage{xcolor}
\usepackage{float}
\usepackage{siunitx}
\usepackage{rotating}
\usepackage{multirow}
\usepackage{listings}
\usepackage{hyperref}
\setmarginsrb{3 cm}{2.5 cm}{3 cm}{2.5 cm}{1 cm}{1.5 cm}{1 cm}{1.5 cm}

\title{Tutorial 3: Logic Synthesis Design Flow with YOSYS} 			% Title
\author{Jianwen Zhu}								% Author
\date{\today}									% Date
\newcommand{\course} {ECE1769H}
\newcommand{\quotes}[1]{``#1''}

\makeatletter
\let\thetitle\@title
\let\theauthor\@author
\let\thedate\@date
\makeatother

\pagestyle{fancy}
\fancyhf{}
\lhead{\course}
\rhead{\thetitle}
\cfoot{\thepage}

\newcommand {\comment}[1]{\iffalse #1 \fi}

\begin{document}

%%%%%%%%%%%%%%%%%%%%%%%%%%%%%%%%%%%%%%%%%%%%%%%%%%%%%%%%%%%%%%%%%%%%%%%%%%%%%%%%%%%%%%%%%

\begin{titlepage}
	\centering
    \vspace*{0.5 cm}
    \includegraphics[scale = 0.12]{UofT.png}\\[1.0 cm]	% University Logo
    \textsc{\LARGE University of Toronto}\\[2.0 cm]	% University Name
	\textsc{\Large \course}\\[0.5 cm]				% Course Code
	\textsc{\large Behavioral Synthesis of Digital Integrated Circuits}\\[0.5 cm]				% Course Name
	\rule{\linewidth}{0.2 mm} \\[0.4 cm]
	{ \huge \bfseries \thetitle}\\
	\rule{\linewidth}{0.2 mm} \\[1.5 cm]
	
	%\begin{minipage}{0.4\textwidth}
		%\begin{flushleft} 
		\large
			\emph{Instructor:}\\
			\theauthor 
		%	\end{flushleft}
		%	\end{minipage}~
			
 
	\vfill
	
\end{titlepage}

%%%%%%%%%%%%%%%%%%%%%%%%%%%%%%%%%%%%%%%%%%%%%%%%%%%%%%%%%%%%%%%%%%%%%%%%%%%%%%%%%%%%%%%%%
\pagebreak

%%%%%%%%%%%%%%%%%%%%%%%%%%%%%%%%%%%%%%%%%%%%%%%%%%%%%%%%%%%%%%%%%%%%%%%%%%%%%%%%%%%%%%%%%

\section{Introduction}

In the previous tutorials you started with a reference application in
C, and learned two different design flows to convert the reference
application into hardware:

\begin{enumerate}
\item Register Transfer Level (RTL) design flow, where you enter your
  design at RTL in Verilog {\bf manually};
  
\item C-based design flow, where you enter your design in C, and then
  use high level synthesis (HLS) to generate a RTL design {\bf
    automatically}.
\end{enumerate}

In both design flows, you used {\bf compiled functional simulation} to
verify your design at the register transfer level.

In this tutorial, you will learn how to perform {\bf logic synthesis}
to bring the RTL design down into logic network. You will use an open
source synthesis tool \quotes{YOSYS}, which is widely used in academic
research.

\section{Environment Setup}

The required files for this tutorial are available for download
through the following command:

\begin{lstlisting}[language=bash]
  $git clone https://github.com/jianwen-zhu/2024-fall-ece1749-jzhu.git
\end{lstlisting}

As it is the same as the previous tutorials, you can simply work
within the same directory cloned from the git repository previously.

The required software (YOSYS) for this tutorial is already installed
on UG machines. Make sure to add the following path variable to
{site.mk} file.

\begin{lstlisting}[language=bash]
YOSYS_PATH = /usr/bin
\end{lstlisting}

\section{Logic Synthesis}

\subsection{Perform Logic Synthesis}

Go to either the dot\_rtl1 or dot_c directory. 

The following make target takes the Verilog RTL as input and synthesize
the design. This is accomplished by first generating a YOSIS synthesis
script, \quotes{synthe.ys} and then use it to drive the YOSIS tool to
perform logic synthesis.

\begin{lstlisting}[language=bash]
  $ make syn
\end{lstlisting}

\subsection{Inspecting Design Statistics}

The following make target reports the design statistics of the
synthesized design from the previous step. The statistics can be found
in the \quotes{synth.stat} file.

\begin{lstlisting}[language=bash]
  $ make stat
\end{lstlisting}

\subsection{Tweaking Synthesis Options}

In your \quotes{common.mk} file, three empty variables has been
defined. The rule checks the the value of these variables and
generates the synthesis scripts based on that. The condition
\quotes{test -n "\$var"} is true if the variable has non-empty value
while the condition \quotes{test -z "\$var"} is the exact
opposite. The defined variables determines whether to:

\begin{itemize}
\item \textbf{YOSYS\_GLOBRST} : Add a global reset to the design.
\item \textbf{YOSYS\_COARSE}: Synthesis using ABC synthesis tool\\(another well-known, widely used, academic synthesis tool).
\item \textbf{YOSYS\_SPLITNETS}: Split multi-bit nets in to single-bit nets.
\end{itemize}

Please refer to the Yosys users' manual for more details.

\section{Exercises}

You are encouraged to go through logic synthesis for both
\quotes{dot-c} (C-based design) and \quotes{dot-rtl1} (manual design)
implementations. The generated statistical synthesis report can be
used to evaluate the design efficiency.

\end{document}
